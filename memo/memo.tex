\documentclass{scrartcl}

\usepackage{amssymb}
\usepackage{amsthm}
\usepackage{amsmath}

\usepackage{algpseudocode}
\usepackage{algorithm}

\newtheorem{definition}{Definition}
\newtheorem{assumption}{Assumption}
\newtheorem{theorem}{Theorem}
\newtheorem{corollary}{Corollary}[theorem]

\newcommand{\FS}{\mathrm{FS}}
\newcommand{\NN}{\mathbb{N}}
\newcommand{\TT}{\mathbb{T}}
\newcommand{\PP}{\mathbb{P}}
\newcommand{\BB}{\mathbb{B}}

\let\oldvdash\vdash
\renewcommand{\vdash}{{\oldvdash}}
\let\oldnvdash\nvdash
\renewcommand{\nvdash}{{\oldnvdash}}

\begin{document}

\section{Preliminaries}

\begin{assumption}
  All functions definable in our logic with Church's Thesis are computable.
\end{assumption}

\section{Formal Systems}

\begin{definition}[Formal Systems]
  A formal system $\FS = (S, \neg, \vdash)$ consists of a type $S : \TT$ of logical sentences, a negation function $\neg : S \to S$ and a provability predicate $\vdash : S \to \PP$ fulfilling the following properties:
  \begin{itemize}
    \item $S$ is enumerable and discrete.
    \item $\vdash$ is enumerable.
    \item $\mathrm{FS}$ is consistent: $\forall s. \neg(\vdash s \land \vdash \neg s)$.
  \end{itemize}
\end{definition}

$\FS$ can easily be instantiated with any reasonable formal logic with respect to a set of axioms, e.g.\ first- or second-order logic with a natural deduction system and Robinson's Q.

\begin{definition}[Completeness]
  A formal system $(S, \neg, \vdash)$ is called complete if every sentence can either be proven or disproven: $\forall s. \vdash s \lor \vdash \neg s$.
\end{definition}
\begin{theorem}[Deep Negations]\label{thm:dn}
  Let $(S, \neg, \vdash)$ be a complete formal system and $s : S$. Then $\vdash \neg s \iff \nvdash s$.
\end{theorem}
\begin{proof}
  $\implies$ by consistency, $\impliedby$ by completeness.
\end{proof}

\begin{theorem}[Co-enumerability of Provability] \label{thm:coenum-prov}
  In any complete formal system $(S, \neg, \vdash)$, $\vdash$ is co-enumerable.
\end{theorem}
\begin{proof}
  Consider the following function $P$:
  % TODO
  sdfmkldsfsd
\end{proof}
\begin{theorem}[Decidability of Provability]\label{thm:dec-prov}
  In any complete formal system $\FS = (S, \neg, \vdash)$, $\vdash$ is decidable.
\end{theorem}
\begin{proof}
  We get enumerability from the definition of $\FS$ and co-enumerability by Theorem~\ref{thm:coenum-prov}. To constructively get a decider from this we need a form of Post's Theorem. We use a formulation by [] that additionally only requires discreteness of $S$ and logical decidability of $\vdash$. We can easily get both by the definition of $\FS$ and by using completeness respectively.
\end{proof}
This result can also be shown by directly giving a decider: Given a sentence $s$ we enumerate all provable sentences $s'$ and check whether $s = s'$ or $\neg s = s'$ using discreteness. Correctness is easy to show and totality is by completeness. Note that this does not require enumerability of $S$. However, this approach is tedious to handle in Coq as it requires working with a total step-indexed function $S \to \NN \to \mathcal{O}(\BB)$. Converting it to a function $S \to \BB$ would, to our knowledge, again require enumerability of $S$.

\section{Weak Representability}
To be able to argue about computability, halting and divergence of untyped higher-order functions in Coq we need a constructive version of the Church-Turing-Thesis by []:
\begin{assumption}
  There is an evaluation function $\varphi_{c} : \NN \rightharpoonup \NN$ such that for every definable function $f : \NN \rightharpoonup \NN$ there is a code $c : \NN$ such that $\forall x. \varphi_{c}(x) = f~x$.
\end{assumption}
We encode partial functions $\NN \rightharpoonup \NN$ in our type theory as step-indexed functions $\NN \to \NN \to \mathcal{O} \NN$. We will keep the conversion between functions and codes implicit, as well as pairing functions between $\NN^{2}$ and $\NN$ to be able to apply the Church-Turing-Thesis to functions of higher arity. % TODO do i even need higher arity?

\begin{definition}[Weak Representability]
  Let $\FS = (S, \neg, \vdash)$ be a formal system, $X : \TT, P : X \to \PP$ be a predicate, $r : X \to S$. $r$ weakly represents $P$ in $\FS$, if $\forall x, P x \iff \vdash r\ x$.

  If such an $r$ exists, we call $p$ representable in $\FS$.
\end{definition}

\begin{theorem}
  Let $\FS = (S, \neg, \vdash)$ be a formal system that can weakly represent the following predicate:
  \begin{align*}
    H\ c\ x := \varphi_{c}(x) \text{ is defined}
  \end{align*}
  Then the Halting Problem is decidable.
\end{theorem}
\begin{proof}
  By Theorem \ref{thm:dec-prov}, $\vdash$ is decidable, and, because decidability transports across equivalences, therefore also $H$.
\end{proof}
\begin{corollary}[Weak Gödel's First Incompleteness Theorem]
  Any formal system that can weakly represent $H$ is incomplete.
\end{corollary}


\section{Representability}

\begin{definition}[Representability]\label{def:repr}
  Let $\FS = (S, \neg, \vdash)$ be a formal system and $r : (\NN \rightharpoonup \NN) \to \NN \to \NN \to S$. $r$ represents all functions in $\FS$, if for every $f$:
  \begin{itemize}
    \item $\forall x y. f~x = y \implies \vdash r~f~x~y$ and
    \item $\forall x y y'. f~x = y \implies y \neq y' \implies \nvdash r~f~x~y'$.
  \end{itemize}
  If such an $r$ exists, $\FS$ is said to represent all functions.
\end{definition}


\begin{definition}[Consistent Guessing]
  A language $L \subseteq \NN$ fulfills consistent guessing if
  $$\{(c, x) \mid \varphi_{c}(x) = 0\} \subset L \quad \land \quad \{(c, x) \mid \varphi_{c}(x) \text{ is defined} \land \varphi_{c}(x) \neq 0\} \cap L = \emptyset,$$
  or equivalently, % TODO notation for holds with another value
  $$\forall c v.  (\varphi_{c}(x) = 0 \implies (c, x) \in L) \quad \land \quad (\varphi_{c}(x) \text{ is defined} \land \varphi_{c}(x) \neq 0 \implies (c, x) \notin L).$$
\end{definition}
Note that this definition does not place any restrictions on tuples $(c, x)$ such that $\varphi_{c}(x)$ is undefined, or rather, such that $c$ does not halt on input $x$.

\begin{theorem}[Consistent guessing is undecidable]
  No language $L \subseteq \NN$ that fulfills consistent guessing is decidable.
\end{theorem}
\begin{proof}
  Let $f : \NN \to \NN \to \{0, 1\}$ be a total function such that $\forall cx. f~c~x = 0 \iff (c, x) \in L$. Choose
  \begin{align*}
    g : \NN \to \NN,  g~c := \begin{cases}
      1 & \text{if } f~c~c = 0 \\
      0 & \text{otherwise}
    \end{cases}
  \end{align*}
  We now have $f~g~g = 0 \iff g~g \neq 0 \iff (g, g) \notin L \iff f~g~g \neq 0$. Therefore, $L$ is undecidable.
\end{proof}

\begin{theorem}[Consistent guessing is decidable]
  Let $\FS = (S, \neg, \vdash)$ be a complete formal system  and $r$ a function that represents all functions in $FS$. Then there is a decidable language that fulfills consistent guessing.
\end{theorem}
\begin{proof}
  Let $h : \NN \to \NN \to \{0, 1\}$ be the following function:
  \begin{align*}
    h(c, x) := \begin{cases}
      0 & \text{if } r~c~x~0 \text{ is provable} \\
      1 & \text{otherwise}
    \end{cases}
  \end{align*}
  $h$ is and total by Theorem~\ref{thm:dec-prov} and completeness. It suffices to show that $L := \{(c, x) \mid h(c, x) = 0\}$ fulfills consistent guessing.

  Let $c, x$ be such that $\varphi_{c}(x) = 0$. By Definition \ref{def:repr} we have $\vdash r~c~x~0$, and therefore $h(c, x) = 0$ and $(c, x) \in L$.

  Let $c, x$ be such that $\varphi_{c}(x) \neq 0$. Similarly, we have $\nvdash r~c~x~0$ and therefore $h(c, x) = 1$ and $(c, x) \notin L$.
\end{proof}

\begin{corollary}
  There is no complete formal system that can represent all functions.
\end{corollary}
\begin{corollary}[Gödel's First Incompleteness Theorem]
  Any formal system that can represent all functions is incomplete.
\end{corollary}

\begin{theorem}[Explicit Incompleteness]
  Let $\FS = (S, \neg, \vdash)$ be a formal system, $r$ a function that represents all functions in $\FS$. Dann



  Then there is a sentence $s$ such that $\nvdash s$ and $\nvdash \neg s$.
  % TODO write down sentence explicitely
\end{theorem}




\end{document}
